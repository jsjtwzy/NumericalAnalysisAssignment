%% 
%% Copyright 2007-2020 Elsevier Ltd
%% 
%% This file is part of the 'Elsarticle Bundle'.
%% ---------------------------------------------
%% 
%% It may be distributed under the conditions of the LaTeX Project Public
%% License, either version 1.2 of this license or (at your option) any
%% later version.  The latest version of this license is in
%%    http://www.latex-project.org/lppl.txt
%% and version 1.2 or later is part of all distributions of LaTeX
%% version 1999/12/01 or later.
%% 
%% The list of all files belonging to the 'Elsarticle Bundle' is
%% given in the file `manifest.txt'.
%% 
%% Template article for Elsevier's document class `elsarticle'
%% with harvard style bibliographic references

\documentclass[preprint,12pt]{elsarticle}

%% Use the option review to obtain double line spacing
%% \documentclass[preprint,review,12pt]{elsarticle}

%% Use the options 1p,twocolumn; 3p; 3p,twocolumn; 5p; or 5p,twocolumn
%% for a journal layout:
%% \documentclass[final,1p,times]{elsarticle}
%% \documentclass[final,1p,times,twocolumn]{elsarticle}
%% \documentclass[final,3p,times]{elsarticle}
%% \documentclass[final,3p,times,twocolumn]{elsarticle}
%% \documentclass[final,5p,times]{elsarticle}
%% \documentclass[final,5p,times,twocolumn]{elsarticle}

%% For including figures, graphicx.sty has been loaded in
%% elsarticle.cls. If you prefer to use the old commands
%% please give \usepackage{epsfig}

%% The amssymb package provides various useful mathematical symbols
\usepackage{amssymb}
\usepackage{graphicx}
\usepackage{subfigure}
%% The amsthm package provides extended theorem environments
%% \usepackage{amsthm}

%% The lineno packages adds line numbers. Start line numbering with
%% \begin{linenumbers}, end it with \end{linenumbers}. Or switch it on
%% for the whole article with \linenumbers.
%% \usepackage{lineno}

\journal{Nuclear Physics B}

\begin{document}

\begin{frontmatter}

%% Title, authors and addresses

%% use the tnoteref command within \title for footnotes;
%% use the tnotetext command for theassociated footnote;
%% use the fnref command within \author or \address for footnotes;
%% use the fntext command for theassociated footnote;
%% use the corref command within \author for corresponding author footnotes;
%% use the cortext command for theassociated footnote;
%% use the ead command for the email address,
%% and the form \ead[url] for the home page:
%% \title{Title\tnoteref{label1}}
%% \tnotetext[label1]{}
%% \author{Name\corref{cor1}\fnref{label2}}
%% \ead{email address}
%% \ead[url]{home page}
%% \fntext[label2]{}
%% \cortext[cor1]{}
%% \affiliation{organization={},
%%             addressline={},
%%             city={},
%%             postcode={},
%%             state={},
%%             country={}}
%% \fntext[label3]{}

\title{}

%% use optional labels to link authors explicitly to addresses:
%% \author[label1,label2]{}
%% \affiliation[label1]{organization={},
%%             addressline={},
%%             city={},
%%             postcode={},
%%             state={},
%%             country={}}
%%
%% \affiliation[label2]{organization={},
%%             addressline={},
%%             city={},
%%             postcode={},
%%             state={},
%%             country={}}

\author{}

\affiliation{organization={},%Department and Organization
            addressline={}, 
            city={},
            postcode={}, 
            state={},
            country={}}

\begin{abstract}
%% Text of abstract

\end{abstract}

%%Graphical abstract
\begin{graphicalabstract}
%\includegraphics{grabs}
\end{graphicalabstract}

%%Research highlights
\begin{highlights}
\item Research highlight 1
\item Research highlight 2
\end{highlights}

\begin{keyword}
%% keywords here, in the form: keyword \sep keyword

%% PACS codes here, in the form: \PACS code \sep code

%% MSC codes here, in the form: \MSC code \sep code
%% or \MSC[2008] code \sep code (2000 is the default)

\end{keyword}

\end{frontmatter}

%% \linenumbers

%% main text
\section{}
\label{}
	\begin{eqnarray}
		L_5(x) = - 0.000237139787679362 x_{1}^{5} + 0.00714565151447801 x_{1}^{4} \\
		- 0.0815449154686002 x_{1}^{3} + 0.436753337136501 x_{1}^{2} - 1.08718447439804 x_{1} + 1.0 \\
		1.4497759011197 \cdot 10^{-7} x_{2}^{10} - 8.04313770346883 \cdot 10^{-6} x_{2}^{9} \\
		L_{10}(x) = + 0.000195169673928753 x_{2}^{8} - 0.00272201830501502 x_{2}^{7} + 0.0241109337940954 x_{2}^{6} \\
		- 0.141420065599032 x_{2}^{5} + 0.555275062775147 x_{2}^{4} - 1.44050898256705 x_{2}^{3} \\
		+ 2.37065457934998 x_{2}^{2} - 2.26557678096194 x_{2} + 1.0 \\
		H_3(x) = - 3.09027727303979 \cdot 10^{-5} x^{6} + 0.00116390879146581 x^{5} \\
		- 0.016758293090973 x^{4} + 0.111599878253313 x^{3} - 0.305039438818523 x^{2} \\
		+ 5.55111512312578 \cdot 10^{-17} x + 1.0 \\
		H_4(x) = 4.36795874184168 \cdot 10^{-7} x^{9} - 2.37088587692207 \cdot 10^{-5} x^{8} \\
		+ 0.000552414443654585 x^{7} - 0.0071880511370789 x^{6} + 0.056741858890937 x^{5} \\
		- 0.274584178076603 x^{4} + 0.770347965032566 x^{3} - 1.0169426021137 x^{2} \\
		+ 1.11022302462516 \cdot 10^{-16} x + 1.0 \\
		s(x) = 0.0317725477299633 x^{3} - 0.61357667740634 x + 1.0, \\
		- 0.0397574264410559 x^{3} + 0.429179845026115 x^{2} - 1.47193636745857 x + 1.57223979336815, \\
		0.0101907016574591 x^{3} - 0.170197692156065 x^{2} + 0.925573781270149 x - 1.6244404049368, \\
		- 0.00273476004282412 x^{3} + 0.0624606184490329 x^{2} - 0.470376082360437 x + 1.16745932232437, \\
		0.000528937096457655 x^{3} - 0.0158681128937297 x^{2} + 0.156253768381664 x - 0.503553612987903 \\
		1.0 x_{1}^{2} + 1.0 x_{1} + 1.0 \\
		2.0 x_{1}^{2} + 3.0 x_{1} + 1.0 \\
		\left[ 1, \  x - \frac{1}{2}, \  x^{2} - x + \frac{1}{6}, \  x^{3} - \frac{3 x^{2}}{2} + \frac{3 x}{5} - \frac{1}{20}, \  x^{4} - 2 x^{3} + \frac{9 x^{2}}{7} - \frac{2 x}{7} + \frac{1}{70}\right] \\
		a_{0} + a_{1} x - \frac{a_{1}}{2} + a_{2} x^{2} - a_{2} x + \frac{a_{2}}{6} + a_{3} x^{3} - \frac{3 a_{3} x^{2}}{2} + \frac{3 a_{3} x}{5} - \frac{a_{3}}{20} + a_{4} x^{4} - 2 a_{4} x^{3} + \frac{9 a_{4} x^{2}}{7} - \frac{2 a_{4} x}{7} + \frac{a_{4}}{70} \\
		2.88528569960203 x^{4} - 12.3348255432603 x^{3} + 16.2747397427789 x^{2} - 5.29872761817569 x + 0.942720952928946 \\
		- 6.62206149153474 x^{3} + 12.8147250886843 x^{2} - 4.65906623876537 x + 0.926587005389723
	\end{eqnarray}
	\begin{figure}
		\centering
		\subfigure[Fig1]{
			\includegraphics[scale=0.6]{figure_3.png} \label{1}
		}
		\quad
		\subfigure[Fig1]{
			\includegraphics[scale=0.6]{figure_4.png} \label{2}
		}
		\quad
		\subfigure[Fig3]{
			\includegraphics[scale=0.5]{figure_5.png} \label{3}
		}
		\quad
	\end{figure}
%% The Appendices part is started with the command \appendix;
%% appendix sections are then done as normal sections
%% \appendix

%% \section{}
%% \label{}

%% For citations use: 
%%       \citet{<label>} ==> Jones et al. [21]
%%       \citep{<label>} ==> [21]
%%

%%\bibliographystyle{myIEEEtran}
%%\bibliography{References}

%% If you have bibdatabase file and want bibtex to generate the
%% bibitems, please use
%%
%%  \bibliographystyle{elsarticle-num-names} 
%%  \bibliography{<your bibdatabase>}

%% else use the following coding to input the bibitems directly in the
%% TeX file.

%%\begin{thebibliography}{00}

%% \bibitem[Author(year)]{label}
%% Text of bibliographic item

%%\bibitem[ ()]{}

%%\end{thebibliography}
\end{document}

\endinput
%%
%% End of file `elsarticle-template-num-names.tex'.
